%%%%%%%% ICML 2025 EXAMPLE LATEX SUBMISSION FILE %%%%%%%%%%%%%%%%%

\documentclass{article}

% Recommended, but optional, packages for figures and better typesetting:
\usepackage{microtype}
\usepackage{multirow}
\usepackage{graphicx}
\usepackage{subcaption}
\usepackage{booktabs} % for professional tables
\usepackage{paralist}
\usepackage[dvipsnames]{xcolor}

% hyperref makes hyperlinks in the resulting PDF.
% If your build breaks (sometimes temporarily if a hyperlink spans a page)
% please comment out the following usepackage line and replace
% \usepackage{icml2025} with \usepackage[nohyperref]{icml2025} above.
\usepackage{hyperref}


% Attempt to make hyperref and algorithmic work together better:
\newcommand{\theHalgorithm}{\arabic{algorithm}}
% Use the following line for the initial blind version submitted for review:
% \usepackage{icml2025}
% Use the following line for the initial blind version submitted for review:
% \usepackage{icml2025}

% If accepted, instead use the following line for the camera-ready submission:
\usepackage[accepted]{icml2025}

% For theorems and such
\usepackage{amsmath}
\usepackage{amssymb}
\usepackage{mathtools}
\usepackage{amsthm}

% if you use cleveref..
\usepackage[capitalize,noabbrev]{cleveref}

\Crefformat{figure}{#2Fig.~#1#3}
\Crefmultiformat{figure}{Figs.~#2#1#3}{ and~#2#1#3}{, #2#1#3}{ and~#2#1#3}

%%%%%%%%%%%%%%%%%%%%%%%%%%%%%%%%
% THEOREMS
%%%%%%%%%%%%%%%%%%%%%%%%%%%%%%%%
\theoremstyle{plain}
\newtheorem{theorem}{Theorem}[section]
\newtheorem{proposition}[theorem]{Proposition}
\newtheorem{lemma}[theorem]{Lemma}
\newtheorem{corollary}[theorem]{Corollary}
\theoremstyle{definition}
\newtheorem{definition}[theorem]{Definition}
\newtheorem{assumption}[theorem]{Assumption}
\theoremstyle{remark}
\newtheorem{remark}[theorem]{Remark}

% Todonotes is useful during development; simply uncomment the next line
%    and comment out the line below the next line to turn off comments
%\usepackage[disable,textsize=tiny]{todonotes}
\usepackage[textsize=tiny]{todonotes}

% darkorchid
\definecolor{darkorchid}{rgb}{0.6, 0.2, 0.7}

% new command for \textbf{RQ\textsubscript{i}}
\newcommand{\RQ}[1]{\textbf{RQ\textsubscript{#1}}}
\newcommand{\fname}{\textsc{SHAC++}}
\newcommand{\fnamer}{\textsc{SHAC+}}
\newcommand{\sentence}[1]{(\textbf{\color{darkorchid}#1})}
% \renewcommand{\sentence}[1]{} % Uncomment to remove sentences

\begin{document}

\twocolumn[
\icmltitle{SHAC++: A Neural Network to Rule All Differentiable Simulators}

\begin{icmlauthorlist}
\icmlauthor{Francesco Bertolotti}{unimi}
\icmlauthor{Gianluca Aguzzi}{unibo}
\icmlauthor{Walter Cazzola}{unimi}
\icmlauthor{Mirko Viroli}{unibo}
\end{icmlauthorlist}

\icmlaffiliation{unimi}{Department of Computer Science, Università di Milano, Milano, Italy}
\icmlaffiliation{unibo}{Department of Computer Science, Università di Bologna, Bologna, Italy}

\icmlcorrespondingauthor{TODO}{todo@todo}

\icmlkeywords{Reinforcement Learning, MachineLearning, World Models, Differentiable simulation}

\vskip 0.3in
]

\printAffiliationsAndNotice{}

\input sects/abstract.tex
\input sects/introduction.tex
\input sects/background.tex
\input sects/shacpp.tex
\input sects/experiments.tex
\input sects/discussion.tex
\input sects/ablation.tex
\input sects/conclusion.tex
\input sects/related.tex
\input sects/acknowledgements.tex
\input sects/statement.tex

\bibliography{local.bib}
\bibliographystyle{icml2025}
\input appendices/appendix.tex

\end{document}

