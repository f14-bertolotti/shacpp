\section{Discussion}
In the following, we discuss the results obtained in \Cref{sect:experiments} and provide some insights into the performance of \fname{} compared to SHAC and PPO trying to answer the research questions posed in \Cref{sect:introduction}.
Note that \fname{} does not aim to outperform SHAC, it merely aims to perform comparably with SHAC while being applicable even in non-differentiable environment. 
\begin{center}
\RQ{1} \textbf{Can we train a neural network to approximate the gradients of a differentiable simulator?}
\end{center}

The proposed framework, \fname{}, appears to be successful in emulating the behavior of SHAC in differentiable environments such as transport and sampling. 
Therefore, we can conclude that it is definitely possible to approximate the gradients of a differentiable environment with a neural network. 
However, we note that this approach, while obtaining comparable results in most setups, it tends to display higher variance. 
Both neural networks representing the reward and transition functions require sufficient training before becoming effective, which may entail multiple episodes. Interestingly, \fname{} outperforms SHAC in the Sampling scenario. A possible explanation is that the transition network focuses on the scenario's nuanced dynamics, whereas SHAC relies on broader environmental dynamics, creating additional gradient noise.
\sentence{GA: francesco verifica che questa discussione può essere allienata con i risultati ottenuti, o magari c'è un'altra spiegazione che mi sono perso?}
\begin{center}
\RQ{2} \textbf{How does our algorithm compare to PPO and SHAC in both single-agent and multi-agent settings?}
\end{center}
The proposed algorithm appears capable of outperforming PPO across all scenarios showing also a better sample efficiency.
In single-agent cases, while PPO manages to find near-good policies (e.g., in the Dispersion scenario), 
it often fails to complete scenarios within the same number of steps. 
Indeed, \fname{} achieves better results in all environments (see \Cref{tab:max-rewards}). 
Despite having to train multiple networks, \fname{} converges much faster to optimal policies, 
as gradient guidance enables quicker and more accurate convergence to the desired outcome. 
This pattern becomes even more evident in multi-agent scenarios. 
In these cases, PPO often struggles to find even marginally good policies. 
\fname{}, on the other hand, eventually succeeds in understanding the environment and subsequently improves the policy to enable cooperation---see both \Cref{tab:max-rewards} and \Cref{fig:experiments}.
\comment{GA: c'è altro che possiamo discutere rispetto a PPO? Forse dobbiamo dire che anche se converge prima è computazionalmente più costoso?}
\begin{center}
\RQ{3} \textbf{How does the performance of these algorithms change as the search space increases?}
\end{center}
TODO
