\section{Background \& Notation}\label{sect:background}

\subsection{Markov Decision Processes}
To provide context for our work, we first introduce the reinforcement learning (RL) framework. Each task is modeled as a finite-horizon Markov Decision Process (MDP) $\mathcal{M} = (\mathcal{S}, \mathcal{A}, F, R, T, \gamma, \mu)$, where $\mathcal{S}$ represents the state space, $\mathcal{A}$ denotes the action space, $F:\mathcal{S} \times \mathcal{A} \rightarrow \mathcal{S}$ is the transition function, $R:\mathcal{S} \times \mathcal{A} \times \mathcal{S} \rightarrow \mathbb{R}$ is the reward function, $\gamma \in [0, 1]$ is the discount factor, and $T \in \mathbb{N}$ is the time horizon. The initial state distribution is denoted by $\mu$.

Further, let $\mathcal{T}$ denote the space of trajectories, defined as $\mathcal{T} = (\mathcal{S} \times \mathcal{A} \times \mathcal{S} \times \mathbb{R})^*$, where each trajectory consists of a sequence of starting states, actions, subsequent states, and rewards. We will use the symbols $\tau=((s_0, a_0, s_1, r_0), (s_1, a_1, s_2, r_1), \ldots)$ to denote a generic trajectory in $\mathcal{T}$. When dealing with multiple trajectories, we will use the notation $s_i^\tau, a_i^\tau$, and $r_i^\tau$.

A policy is a function $\pi:\mathcal{S}\rightarrow\mathcal{A}$ that maps states to actions. We can use the policy to sample trajectories from the MDP by iteratively applying the transition and reward function and the policy. In this case, we have that the $i$-th element of a trajectory is $(s_i, a_i, s_{i+1}, r_i) = (s_i, \pi(s_i), F(s_i, a_i), R(s_i,a_i,F(s_i, a_i))$. The initial state $s_0$ is sampled from the initial state distribution $\mu$.

For brevity, we will simply denote with $\tau\sim\pi$ a trajectory sampled by following the policy $\pi$.

The goal of the MDP is to find a policy that maximizes the expected return 
$$\mathbb{E}_{\tau\sim\pi}\left[\sum_{t=0}^{T-1}\gamma^t r_t^\tau \right]$$

\subsection{Neural Networks}
A neural network is a function $f_{\theta}:\mathbb{R}^n\rightarrow\mathbb{R}^m$ parameterized by $\theta$ that maps an input vector $x\in\mathbb{R}^n$ to an output vector $y\in\mathbb{R}^m$. The parameters $\theta$ of the network are learned by minimizing a loss function $\mathcal{L}_\theta$ with respect to the training data.

In this work, we will often approximate functions, such as transition ($F$) or reward ($R$), with neural network. Therefore, for each function $g$, we will denote the corresponding neural network with $g_\theta$. For instance, the transition and its corresponding neural network are denoted as $F$ and $F_{\theta}$, respectively. The corresponding loss will be denoted as $\mathcal{L}_\theta^g$ and it will aim to minimize the differences between $g$ and $g_\theta$.

While we use a unique set of parameters $\theta$ for all the networks. Each network can either use its own subset of $\theta$ or it can share parameters with other networks. In practice, the parameters will never be shared.

\subsection{SHAC}

To solve an MDP $\mathcal{M}$, SHAC trains a policy $\pi_\theta$ by minimizing the loss function $\mathcal{L}_\theta^{\pi}:\mathcal{T}^N\rightarrow\mathbb{R}$ for a batch of $N$ trajectories $B = \{\tau_1, \ldots, \tau_N\}$ sampled from the MDP. The loss is defined as:  
\begin{equation}\label{eq:shac}
    \mathcal{L}_\theta^{\pi}(B) = \frac{1}{NH}\sum_{\tau\in B} \sum_{t=t_0}^{t_0+H-1} \gamma^{t-t_0} R(s_t^\tau, a_t^\tau) + \gamma^H V_\theta(s^\tau_{t_0+T})
\end{equation}
where $N$ is the number of sampled trajectories, $H \ll T$ is a time horizon, $s_t^\tau$ is the state at time $t$ for trajectory $\tau$, $a_t^\tau$ is the action taken at time $t$ for trajectory $\tau$, and $V(s): \mathcal{S} \rightarrow \mathbb{R}$ is the value function estimating the expected return starting from state $s$. From Equation~\ref{eq:shac}, it is apparent that SHAC assumes that the policy, reward, value, and transition functions are all differentiable, i.e., $\pi_\theta,V_\theta,F,R \in \mathcal{C}^1$. While this is an common requirement for policy and value functions, differentiable transition and reward functions are rarely available. 

The value function $V_\theta$ is trained to approximate a temporal difference-$\lambda$ formulation \cite{Sutton98} $V$, which in SHAC translates to minimize the loss $\mathcal{L}_\theta^{V}:\mathcal{T}^N\rightarrow\mathbb{R}$.

$$ \mathcal{L}_\theta^V(B)=\frac{1}{NT}\sum_{\tau\in B}\sum_{t=0}^T\left|\left| V_\theta(s^\tau_t) - V(s^\tau_t) \right|\right|^2 $$
$$ V(s_t) = (1-\lambda) \left(\sum_{k=1}^{h-t-1}\lambda^{k-1}G_t^k\right) + \lambda^{h-t-1}G_t^{h-t}$$
$$ G_t^k = \left(\sum_{l=0}^{k-1}\gamma^l r_{t+l}\right) + \gamma^k V(s_{t+k})$$

$G^t_k$ represents the discounted sum of rewards up to step $k$, plus the bootstrapped value of the state at step $t+k$. The estimates $V$ is computed by discounting longer (higher $k$) discounted sums, $G^k_t$.

Finally, Figure~\ref{fig:shac} presents an unrolled view of the SHAC algorithm.
