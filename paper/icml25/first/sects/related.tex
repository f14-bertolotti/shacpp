\section{Related Work}

\paragraph{Differentiable Engines}
Game engines~\cite{Bellemare13,Balla23} and Physics engines \cite{Todorov12,Coumans16} have been widely used in the field of policy learning. Recently, there has been a trend towards differentiable engines, which allows for the use of gradient-based optimization methods, such as backpropagation. Notable examples are \cite{Freeman21,Howell22,Genesis24}. These frameworks may be built on top of auto differentiation libraries such \cite{Bradbury18,Ansel24} (this is the case of VMAS) or directly with analytical gradients \cite{Carpentier18,Werling21}. Further, several works tackle the challenge of the non-smoothness of the contact dynamics \cite{Degrave19,Moritz20}

\paragraph{Deep Reinforcement Learning}
Deep reinforcement learning has revolutionized policy learning tasks by combining reinforcement learning with deep neural networks. 
Foundational works like DQN~\cite{Mnih13} (and its variants)
and AlphaZero~\cite{Silver17} demonstrated human-level performance in complex environments such as Atari games and chess. 
Over the years, several algorithms have been proposed to create more stable and efficient learning methods. 
Notable examples include A3C for asynchronous learning~\cite{DBLP:conf/icml/MnihBMGLHSK16}, 
DDPG and PPO for continuous action spaces~\cite{DBLP:journals/corr/LillicrapHPHETS15,Schulman17}, 
and SAC for continuous action spaces with off-policy learning~\cite{DBLP:conf/icml/HaarnojaZAL18}. 
Following the initial era of model-free algorithms, 
model-based algorithms have been introduced to enhance sample efficiency. 
Early works such as Dreamer~\cite{DBLP:conf/iclr/HafnerLB020} and PlaNet~\cite{DBLP:conf/icml/HafnerLFVHLD19} have demonstrated promising results in terms of both sample efficiency and performance.
\paragraph{Multi-Agent Reinforcement Learning}

\paragraph{Reinforcement Learning on Differentiable Engines}


