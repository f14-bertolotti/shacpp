\section{Related Work}

\paragraph{Differentiable Engines}
Game engines~\cite{Bellemare13,Balla23} and Physics engines \cite{Todorov12,Coumans16} have been widely used in the field of policy learning. Recently, there has been a trend towards differentiable engines, which allows for the use of gradient-based optimization methods, such as backpropagation. Notable examples are \cite{Freeman21,Howell22,Genesis24}. These frameworks may be built on top of auto differentiation libraries such \cite{Bradbury18,Ansel24} (this is the case of VMAS) or directly with analytical gradients \cite{Carpentier18,Werling21}. Further, several works tackle the challenge of the non-smoothness of the contact dynamics \cite{Degrave19,Moritz20}

\paragraph{Deep Reinforcement Learning}
Deep reinforcement learning has revolutionized policy learning tasks by combining reinforcement learning with deep neural networks. 
Foundational works like DQN~\cite{Mnih13} (and its variants~\cite{DBLP:conf/aaai/HesselMHSODHPAS18})
and AlphaZero~\cite{Silver17} demonstrated human-level performance in complex environments such as Atari games and chess. 
Over the years, several algorithms have been proposed to create more stable and efficient learning methods. 
Notable examples include A3C for asynchronous learning~\cite{DBLP:conf/icml/MnihBMGLHSK16}, 
DDPG and PPO for continuous action spaces~\cite{DBLP:journals/corr/LillicrapHPHETS15,Schulman17} (also used in modern chatbot like ChatGPT), 
and SAC for continuous action spaces with off-policy learning~\cite{DBLP:conf/icml/HaarnojaZAL18}. 
Following the initial era of model-free algorithms, 
model-based algorithms have been introduced to enhance sample efficiency. 
Early works such as Dreamer~\cite{DBLP:conf/iclr/HafnerLB020} and PlaNet~\cite{DBLP:conf/icml/HafnerLFVHLD19} have demonstrated promising results in terms of both sample efficiency and performance.

\paragraph{Multi-Agent Reinforcement Learning}
Multi-Agent Reinforcement Learning (MARL) is a subfield of reinforcement learning that focuses on learning policies for multiple agents that interact with each other~\cite{albrecht2024multi}. 
This field has a wide range of applications, 
including robotics~\cite{DBLP:journals/air/ChungFYN24}, 
traffic control~\cite{DBLP:journals/tits/ChuWCL20}, 
and game playing~\cite{DBLP:conf/iclr/BakerKMWPMM20,jaderberg2019human}, 
and involves different types of agents such as cooperative, competitive, and mixed. 
In this paper, we primarily focus on cooperative multi-agent scenarios, 
where agents need to collaborate to achieve a common goal. 
Prominent works in this area include extensions of single-agent algorithms to multi-agent settings, 
such as MADDPG~\cite{DBLP:conf/nips/LoweWTHAM17} and MAPPO~\cite{DBLP:conf/nips/YuVVGWBW22}, 
as well as novel solutions to unique challenges in multi-agent settings. Examples of these solutions are COMA~\cite{DBLP:conf/aaai/FoersterFANW18} for multi-agent credit assignment, 
QMIX~\cite{DBLP:conf/icml/RashidSWFFW18} for coordination in cooperative multi-agent settings, 
and Mean Field MARL~\cite{DBLP:conf/icml/YangLLZZW18} for handling a large number of agents.

\paragraph{Reinforcement Learning on Differentiable Engines}


